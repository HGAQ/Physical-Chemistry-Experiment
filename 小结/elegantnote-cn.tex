%!TEX program = xelatex
\documentclass[cn,hazy,pku,12pt,normal,math=newtx,cite=super]{elegantnote}
\title{物理化学实验小结}

\author{刘松瑞 \quad 2100011819 \\ 组号:24 \quad 组内编号:5}
\institute{化学与分子工程学院}


\usepackage{gensymb}
\usepackage{array}
\usepackage{subfigure}
\usepackage[fontset=windows]{ctex}
\usepackage{graphicx}
\usepackage{float}
\usepackage{caption}
\usepackage{multirow}
%\usepackage{subfig}
%\usepackage{float}
\begin{document}

\maketitle


\section{实验小结}

在这次物理化学综合实验中,我们涉及了偶极矩、紫外光谱仪搭建、磁化率、燃烧热、表面吸附、溶解热、蔗糖转化、饱和蒸气压、胶体制备性质、双液系和电化学反应等十一个实验。通过这些实验,我不仅深入了解了物理化学领域的多个方面,而且学到了丰富的实验技能和科学方法。

\subsection{仪器原理与使用方法}

实验涉及了各种仪器,如紫外光谱仪、密度测量仪等,让我深刻理解了其原理和操作方法。这为我今后的实验工作提供了坚实的基础,并拓宽了我的实验技能。

\subsection{软件运用}

学习了多种科学软件,包括Origin、Gaussian等,为数据处理和分析提供了强大的支持。尤其是通过使用 \LaTeX 语言,我能够规范而美观地撰写实验报告,提高了科学文献写作的水平。


\subsection{实验思想}

通过实验,我感悟到了从不可测量量到可测量量的转换。通过使用仪器进行测量和分析,我们能够获取实验数据,进而揭示物质性质的奥秘。
此外,通过实验数据的误差分析,有助于我分析如何提高实验结果的可靠性,也使我能够更加谨慎地处理实验中的各种因素。


这次实验让我全面认识了物理化学的广度和深度,提高了实验设计和数据分析的能力。未来,我将继续深入学习物理化学理论,运用所学知识解决实际问题,为科学研究和应用领域做出更多贡献。
这次实验不仅拓宽了我的知识面,也培养了我的实验操作和科学研究能力。期待在未来的学习和研究中,能够运用这些经验,不断进步。

\end{document}
